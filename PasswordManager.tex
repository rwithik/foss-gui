\documentclass[10pt,a4paper,titlepage]{report}
\usepackage[utf8]{inputenc}
\usepackage{amsmath}
\usepackage{amsfonts}
\usepackage{amssymb}
\usepackage{graphicx}
\usepackage{xcolor}
\usepackage{minted}


\setlength{\arrayrulewidth}{0pt}
\setlength{\tabcolsep}{18pt}

\nonstopmode
\begin{document}

\newpage
\begin{center}
		\textbf{Project Report}\\\vspace{0.4cm}
on\\\vspace{1cm}
\begin{LARGE}
		Password Manager Application
\end{LARGE}\\\vspace{0.4cm}
\textit{GTK3 with Python 3}\vspace{1cm}
\\
\textbf{Computer Science and Engineering}\\\vspace{1cm}
Submitted by\\\vspace{1cm}
\begin{tabular}{l|r}
	\textbf{Roll No.}&\textbf{Name}&
	38& Justine Biju&
	53& Rwithik Manoj&
\end{tabular}
\vfill
\textsc{College of Engineering, Trivandrum\\Semester 4}
\end{center}
\newpage
\tableofcontents

\chapter{Introduction}
\newline\par GTK, or the GIMP Toolkit, is a multi-platform toolkit for creating graphical user interfaces. Offering a complete set of widgets, GTK is suitable for projects ranging from small one-off tools to complete application suites.
\newline\par GTK is written in C but has been designed from the ground up to support a wide range of languages, not only C/C++. Using GTK from languages such as Perl and Python (especially in combination with the Glade GUI builder) provides an effective method of rapid application development.
\newline\par GTK is free software and part of the GNU Project. However, the licensing terms for GTK, the GNU LGPL, allow it to be used by all developers, including those developing proprietary software, without any license fees or royalties.
\newline\par Some popular GTK applications include:
\begin{enumerate}
		\item GIMP
		\item Kitty Terminal
		\item Inkscape
		\item Pigdin
		\item Glade Interface Designer
		\item GNOME Boxes
		\item GNOME Terminal
\end{enumerate}

\chapter{About the Application}
\newline\par The project that we decided to do is a simple password manager. An application that stores your passwords and can place them into your clipboard when you need to use them. Since not everyone should have access to your passwords, it asks for the master password when you start it up and gives you access only if you manage to get the master password correct. 
\newline\par The application stores the service or website, the username(can store multiple usernames for the same website), and password for each username. The user can then copy the password or delete an entry all together. 
\newline\par The github link of our project is https://github.com/rwithik/foss-gui

\chapter{Development of the Application}
\newline\par The code is divided into two files: app.py and windowClass.py. The app.py file generates the initial password verification window. It loads the main application window if the entered password is correct. 
\newline\par We have used the Notify class to send notifications via the system dbus. Messages likeIncorrect password and successfully adding and removing entries are sent as notifications. 
\newline\par The entries are stored in the following format in a dictionary:
\begin{minted}[tabspace=4]{python}
{
website: {
	username: password, 
	username: password, 
	...
	}, 
website: {
	username: password, 
	username: password
	...
	}, 
	...
}
\end{minted}

\section{windowClass.py}
\newline\par This file has a class named MainWindow that inherits the Gtk.Window class. It generates the main application window.
\newline\par The functions defened in the class:
\subsection{add\_new\_entry()}
\newline\par This function is called when the Add entry button is clicked. It inserts a new entry into the dictionary and writes it to the file. It then redraws the whole window. 
\subsection{delete\_clicked()}
\newline\par This function is called when the delete button of any entry is clilcked. The website and username is passed to the function and the corresponding entry in the dictionary is deleted.
\subsection{get\_grid()}
\newline\par This function returns a Gtk.Grid element with all the entries and input fields. 
\subsection{on\_copy\_clicked()}
\newline\par This function is called when the Copy button of an entry is clicked. It shows a notification and places the password into the clipboard, ready to be pasted.
\subsection{send\_notification()}
\newline\par An utility function to display a notification message.
\subsection{\_\_redraw()}
\newline\par A private function that redraws the window after each update. 

\section{app.py}
\newline\par This file import the windowClass.py file. This file generates the master password dialog box. It has only one method defined, verify(), to verify the given password. If the password is correct, it creates a MainWindow object and start the main application window. 



\chapter{Conclusion}
\newline\par The Password Manager is working satisfactorily. We learnt a lot of new things about GTK by completing this project, not only about GTK, but also about the common coding conventions. 

\chapter{References}
\begin{itemize}
		\item stackoverflow.com
		\item GTK Read the Docs
		\item GNOME GTK3 Reference
		\item programcreek.com
\end{itemize}
\end{document}
